\documentclass[12pt,letterpaper]{article}
\usepackage{fullpage}
\usepackage[top=2cm, bottom=4.5cm, left=2.5cm, right=2.5cm]{geometry}
\usepackage{amsmath,amsthm,amsfonts,amssymb,amscd}
\usepackage{lastpage}
\usepackage{enumerate}
\usepackage{fancyhdr}
\usepackage{mathrsfs}
\usepackage{xcolor}
\usepackage{graphicx}
\usepackage{listings}
\usepackage{hyperref}
\usepackage{multicol}

\usepackage{enumitem}


\hypersetup{%
  colorlinks=true,
  linkcolor=blue,
  urlcolor=cyan,
  linkbordercolor={0 0 1}
}
 
\renewcommand\lstlistingname{Algorithm}
\renewcommand\lstlistlistingname{Algorithms}
\def\lstlistingautorefname{Alg.}

\lstdefinestyle{Python}{
    language        = Python,
    frame           = lines, 
    basicstyle      = \footnotesize,
    keywordstyle    = \color{blue},
    stringstyle     = \color{green},
    commentstyle    = \color{red}\ttfamily
}

\setlength{\parindent}{0.0in}
\setlength{\parskip}{0.05in}

% Edit these as appropriate
\newcommand\course{PHYS 243}
\newcommand\hwnumber{3}                  % <-- homework number
\newcommand\MyName{TA: Abtin Shahidi}           % <-- My name

\pagestyle{fancyplain}
\headheight 35pt
\lhead{\MyName}

\chead{\textbf{\Large Mid-term}}
\rhead{\course \\ Deadline:  05/19/2019, 11:59 pm}
\lfoot{}
\cfoot{}
\rfoot{\small\thepage}
\headsep 1.5em

\begin{document}
\section*{Data sets:}
You can choose any of the following data-sets:

\begin{enumerate}
\item Climate Model Simulation Crashes Data Set 

\item Acute Inflammations Data Set 

\item Teaching Assistant Evaluation Data Set 


\item Wine Quality Data Set 
\end{enumerate}

There is a readme.data file in each data set directory. You can find all the extra but relevant information about each data set. (This is on top of original information file)

\section*{Expectation}
\begin{enumerate}
\item Understanding and explaining the data set.

\item Processing data, cleaning up. 

\item Dividing your data into a training and test set. 

\item Choosing the relevant algorithm.

\item Writing a python code to perform learning. (You can reuse every code from the lectures)

\item Evaluating your learning performance. 

\item Making sure your results does not depend on your choosing parameters.
\end{enumerate}



\end{document}


