\documentclass[12pt,letterpaper]{article}
\usepackage{fullpage}
\usepackage[top=2cm, bottom=4.5cm, left=2.5cm, right=2.5cm]{geometry}
\usepackage{amsmath,amsthm,amsfonts,amssymb,amscd}
\usepackage{lastpage}
\usepackage{enumerate}
\usepackage{fancyhdr}
\usepackage{mathrsfs}
\usepackage{xcolor}
\usepackage{graphicx}
\usepackage{listings}
\usepackage{hyperref}

\hypersetup{%
  colorlinks=true,
  linkcolor=blue,
  linkbordercolor={0 0 1}
}
 
\renewcommand\lstlistingname{Algorithm}
\renewcommand\lstlistlistingname{Algorithms}
\def\lstlistingautorefname{Alg.}

\lstdefinestyle{Python}{
    language        = Python,
    frame           = lines, 
    basicstyle      = \footnotesize,
    keywordstyle    = \color{blue},
    stringstyle     = \color{green},
    commentstyle    = \color{red}\ttfamily
}

\setlength{\parindent}{0.0in}
\setlength{\parskip}{0.05in}

% Edit these as appropriate
\newcommand\course{PHYS 243}
\newcommand\hwnumber{1}                  % <-- homework number
\newcommand\MyName{TA: Abtin Shahidi}           % <-- My name

\pagestyle{fancyplain}
\headheight 35pt
\lhead{\MyName}

\chead{\textbf{\Large Homework \hwnumber}}
\rhead{\course \\ Deadline: June 29, 2019}
\lfoot{}
\cfoot{}
\rfoot{\small\thepage}
\headsep 1.5em

\begin{document}

Welcome again to the \textbf{Foundation of applied machine learning}! 

\section*{Problem 1: Python programming}
This is a problem to check your coding skills! So \textbf{do not use} fancy modules and scripts!  

\subsection*{Production of Fibonacci Sequence!}
I have been assigned to write a function that takes an integer value $n$, where $n\in [0,1,2,3,...]$ and return the $n^{th}$ value in Fibonacci sequence. Since, I am very lazy I just wrote the function \textbf{recursively} as you can see below. Another person, who was assigned to do the same thing, wrote the function with \textsc{for loops} and claims his code is much faster!! 

    \lstset{caption={Recursive Fibonacci Function}}
    \lstset{label={lst:alg1}}
     \begin{lstlisting}[style = Python]
     def Fib_rec(n=0):
     if n==0:
         return 1
     elif n==1:
         return 1
     else:
         return Fib_rec(n-1)+Fib_rec(n-2)
    \end{lstlisting}


\begin{enumerate}
  \item
   \textbf{Part 1:} Rewrite the function with for loop.
  \item
   \textbf{Part 2:} Which function is actually faster? (Explain without running the codes)
  \item
   \textbf{Part 3:} Write a code to time the \textbf{average} time for $k$ times function call. A function that takes three arguments (function to time (Fib\_rec), input of the function ($n$), number of runs($k$)) and run the Fib\_rec function $k$ times for the input of $n$ and returns the average time.
   \textit{\textbf{Tip:} You can use the time module in the python:}
   \lstset{caption={importing time modules}}
   \lstset{label={lst:alg1}}
    \begin{lstlisting}[style = Python]
    import time
    # if you run this, the current time in (s) will be recorded in  x
    x=time.time() 
    \end{lstlisting}
    
    \lstset{caption={Timer Function}}
   \lstset{label={lst:alg3}}
    \begin{lstlisting}[style = Python]
    def timer(n, k, f=Fib_rec):
    	<your code>
    	return average_time
    \end{lstlisting}
  \item
   \textbf{Part 4:} Make a plot in which the $x$-axis is the value of the input function $n$ and the $y$-axis is the average time (output of the previous function), for both recursive and non-recursive Fibonacci. (Both in the same plot; also use matplotlib package for making the plots) 
\end{enumerate}


\section*{Problem 2: Linear Algebra}

Given the Matrix below answer the questions:

\begin{equation*}
   M = \begin{pmatrix} 
1 & -4 & 2 \\
-4 & 1 & -2 \\
2 & -2 & -2 
\end{pmatrix}
\end{equation*}


    \item \textbf{Part 1:} Find the determinant, transpose, inverse(if exist) for $M$.

    \item \textbf{Part 2:} Find the eigenvalues and eigenvectors for $M$. 
      
    \item \textbf{Part 3:} Find the Gradient if the $\nabla_A f(A)$ for the following:
   \begin{equation*}
	A=\begin{pmatrix} 
	x_{11} & x_{12} & x_{13} \\
	x_{21} & x_{22} & x_{23} \\
	x_{31} & x_{32} & x_{33} 
	\end{pmatrix}
  \end{equation*}
  \begin{equation*}
   f(A)= x_{11}^2 x_{22} x_{23} + x_{11}x_{12}x_{13}x_{31}-x_{33}^2 x_{32} x_{21}
   \end{equation*}

	\item \textbf{Part 4:} Find the Hessian Matrix for:
   \begin{equation*}
	g(x,y,z)=x^3y+yz\sin(x)+xy^2z^5
  \end{equation*}
   	
\section*{Problem 3: Machine Learning}
 \item \textbf{Part 1:} Explain the difference between validation and test samples. 

 \item \textbf{Part 2:} Explain the difference between supervised and unsupervised learning algorithm. 


\end{document}