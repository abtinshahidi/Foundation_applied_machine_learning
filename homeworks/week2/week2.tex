\documentclass[12pt,letterpaper]{article}
\usepackage{fullpage}
\usepackage[top=2cm, bottom=4.5cm, left=2.5cm, right=2.5cm]{geometry}
\usepackage{amsmath,amsthm,amsfonts,amssymb,amscd}
\usepackage{lastpage}
\usepackage{enumerate}
\usepackage{fancyhdr}
\usepackage{mathrsfs}
\usepackage{xcolor}
\usepackage{graphicx}
\usepackage{listings}
\usepackage{hyperref}

\hypersetup{%
  colorlinks=true,
  linkcolor=blue,
  linkbordercolor={0 0 1}
}
 
\renewcommand\lstlistingname{Algorithm}
\renewcommand\lstlistlistingname{Algorithms}
\def\lstlistingautorefname{Alg.}

\lstdefinestyle{Python}{
    language        = Python,
    frame           = lines, 
    basicstyle      = \footnotesize,
    keywordstyle    = \color{blue},
    stringstyle     = \color{green},
    commentstyle    = \color{red}\ttfamily
}

\setlength{\parindent}{0.0in}
\setlength{\parskip}{0.05in}

% Edit these as appropriate
\newcommand\course{PHYS 243}
\newcommand\hwnumber{2}                  % <-- homework number
\newcommand\MyName{TA: Abtin Shahidi}           % <-- My name

\pagestyle{fancyplain}
\headheight 35pt
\lhead{\MyName}

\chead{\textbf{\Large Homework \hwnumber}}
\rhead{\course \\ Deadline:  July 6, 2019, 11:59 pm}
\lfoot{}
\cfoot{}
\rfoot{\small\thepage}
\headsep 1.5em

\begin{document}
\section*{Problem 1: Find the distribution!}
These are the data-set for the number of car accidents in the rush hour of a small city. You are assigned to predict the \textbf{number of car accidents in a given hour of the day}. You need to find the full probability distribution of this quantity. 
\\

Also, make sure to clearly \textbf{state the assumptions} you are making at each step.

\begin{table}[htp]
\centering
\begin{tabular}{|l|l|l|l|l|l|l|l|l|l|l|}
\hline
16 & 24 & 16 & 12 & 16 & 11 & 14 & 15 & 9 & 14 & 7 \\ \hline
\end{tabular}
\caption{The Number of Accident during rush hour}
\end{table}


\textit{\textbf{Tip:} You can/should make reasonable assumption about the data.} 


\section*{Problem 2: Find the parameters!}
Write a python function to find $\mu$ and $\sigma$:
\begin{equation*}
N(x|\mu,\sigma)=N(x|\mu_1,\sigma_1)N(x|\mu_2,\sigma_2)...N(x|\mu_N,\sigma_N)
\end{equation*}

In which $N$ is a Normal distribution:
\begin{equation*}
N(x|\mu,\sigma)=\frac{1}{\sqrt{2\pi \sigma^2}}\exp(-\frac{(x-\mu)^2}{2\sigma^2})
\end{equation*}

Your function should take two vectors: 
\begin{equation*}
\begin{array}[1]
2 \vec{\mu} \textrm{\texttt{= numpy.array}}[\mu_1, \mu_2, ..., \mu_N] \\
 \vec{\sigma} \textrm{\texttt{= numpy.array}}[\sigma_1, \sigma_2, ...,\sigma_N]
\end{array}
\end{equation*}

And return $\mu$ and $\sigma$.
\end{document}